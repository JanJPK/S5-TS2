%----------------------------------------------------------------------------------------
%	PACKAGES AND DOCUMENT CONFIGURATIONS
%----------------------------------------------------------------------------------------
\documentclass{article}
%\documentclass[tikz, border=2mm]{standalone}

\usepackage{siunitx} % Provides the \SI{}{} and \si{} command for typesetting SI units
\usepackage{graphicx} % Required for the inclusion of images
\usepackage{natbib} % Required to change bibliography style to APA
\usepackage{amsmath} % Required for some math elements 
\setlength\parindent{12pt} % Removes all indentation from paragraphs
%\usepackage{times} % Uncomment to use the Times New Roman font

\usepackage[export]{adjustbox}

\usepackage[utf8]{inputenc} % Język polski
\usepackage{polski}
\usepackage[polish]{babel}

\usepackage[top=1in, bottom=1.25in, left=1.25in, right=1.25in]{geometry} % Marginesy
\usepackage{listings} % Kod programu
\usepackage{indentfirst} % Wcięcie przy pomocy \par
\usepackage{multicol} % Kilka kolumn dla itemize
\usepackage[table,xcdraw]{xcolor} % Dla tabel
\usepackage{float} % Blokowanie tabeli przed przemieszczeniem
%\usepackage{karnaugh-map} % Karnaugh

%\renewcommand{\labelenumi}{\alph{enumi}.} % Zamienia litery w enumerate na a, b, c, ...



%----------------------------------------------------------------------------------------
%	DOCUMENT INFORMATION
%----------------------------------------------------------------------------------------

\title{Technologie Sieciowe 2 - Projekt \\ \textit{Projekt sieci lokalnej dla biura projektowego}} % Title

\author{Jan \textsc{Pajdak} \\ Wojciech \textsc{Słowiński}} % Author name

\date{05.11.2017} % Date for the report
%\date{\today} % Date for the report

\begin{document}

\maketitle % Insert the title, author and date

\begin{center}
\begin{tabular}{l r}
%Data: <data> \\ % Date the experiment was performed
%Prowadzący: & Mgr inż. Jan Paweł II % Instructor/supervisor
Prowadzący: & Dr inż. Przemysław Ryba \\
Termin zajęć: & Wtorek 9:15 TP \\
Grupa: & 3 
%Prowadzący: & Dr inż. Tomasz Babczyński
%Prowadzący: & Dr inż. Tomasz Walkowiak
%Prowadzący: & Dr inż. Jarosław Sugier
%Prowadzący: & Mgr inż. Szymon Datko
\end{tabular}
\end{center}

\tableofcontents % Spis treści

% If you wish to include an abstract, uncomment the lines below
% \begin{abstract}
% Abstract text
% \end{abstract}

%----------------------------------------------------------------------------------------
%	SECTION 0
%----------------------------------------------------------------------------------------

%\begin{multicols}{2}
%	\begin{itemize}
%		\item 
%	\end{itemize}
%\end{multicols}

%\newpage
%\section{Tytuł}} 
%\subsection{Tytuł}
%\par Tekst

%\begin{itemize}
%	\item\textit{Tekst} - Opis
%\end{itemize}

%\begin{enumerate}
%	\item Opis
%\end{enumerate}

%\begin{center}
%	\includegraphics[scale=0.6, center]{nazwa_pliku}
%\end{center}

%\begin{lstlisting}[basicstyle=\small]
%	kod programu;
%\end{lstlisting}

%----------------------------------------------------------------------------------------
%	SECTION 1
%----------------------------------------------------------------------------------------
\newpage
\section{Wstęp}

%----------------------------------------------------------------------------------------
%	SECTION 2
%----------------------------------------------------------------------------------------
\newpage
\section{Inwentaryzacja zasobów}

\begin{table}[H]
	\centering
	\caption{Liczba użytkowników}
	%\label{my-label}
	\begin{tabular}{|l|l|l|l|l|l}
		\hline
		\textbf{Budynek}       & \multicolumn{2}{c|}{1} & \multicolumn{3}{c|}{2}            \\ \hline
		\textbf{Piętro}        & 1          & 2         & 1  & 2  & \multicolumn{1}{l|}{3}  \\ \hline
		\textbf{Konstruktorzy} & 41         & 51        & 12 & 50 & \multicolumn{1}{l|}{0}  \\ \hline
		\textbf{Architekci}    & 48         & 54        & 43 & 20 & \multicolumn{1}{l|}{49} \\ \hline
		\textbf{Projektanci}   & 33         & 17        & 18 & 2  & \multicolumn{1}{l|}{8}  \\ \hline
		\textbf{Zarząd}        & 6          & 24        & 1  & 8  & \multicolumn{1}{l|}{12} \\ \hline
	\end{tabular}
\end{table}

\begin{table}[H]
	\centering
	\caption{Punkty dystrybucyjne}
	%\label{my-label}
	\begin{tabular}{|l|l|l|}
		\hline
		\multicolumn{1}{|c|}{\textbf{Oznaczenie}} & \multicolumn{1}{c|}{Lokalizacja} & \multicolumn{1}{c|}{\begin{tabular}[c]{@{}c@{}}Podłączone \\ punkty abonenckie\end{tabular}} \\ \hline
		\textbf{MDF}                              & B2-P3                            & B2-P3                                                                                        \\ \hline
		\textbf{IDF1}                             & B2-P2                            & B2-P1/2                                                                                      \\ \hline
		\textbf{IDF2}                             & B1-P2                            & B1                                                                                           \\ \hline
	\end{tabular}
\end{table}

%----------------------------------------------------------------------------------------
%	SECTION 3
%----------------------------------------------------------------------------------------
\newpage
\section{Analiza potrzeb użytkowników}

\begin{table}[H]
	\centering
	\caption{Wymagania dot. przepływów między pracownikami a serwerami}
	%\label{my-label}
	\begin{tabular}{|l|l|l|l|l|l|l|l|l|}
		\hline
		\multicolumn{1}{|c|}{\textbf{\begin{tabular}[c]{@{}c@{}}Grupa robocza\\ lub serwer\end{tabular}}} & \multicolumn{2}{c|}{Serwer 1} & \multicolumn{2}{c|}{Serwer 2} & \multicolumn{2}{c|}{Serwer 3} & \multicolumn{2}{c|}{Drukarka} \\ \hline
		\textbf{Konstruktorzy}                                                                            & 500           & 650           & 0             & 0             & 250           & 950           & 10            & 100           \\ \hline
		\textbf{Architekci}                                                                               & 150           & 500           & 650           & 150           & 500           & 500           & 10            & 110           \\ \hline
		\textbf{Projektanci}                                                                              & 0             & 0             & 650           & 500           & 200           & 450           & 10            & 100           \\ \hline
		\textbf{Zarząd}                                                                                   & 0             & 0             & 0             & 0             & 50            & 400           & 10            & 180           \\ \hline
		\textbf{WiFi}                                                                                     & 100           & 150           & 100           & 250           & 200           & 150           & 10            & 120           \\ \hline
	\end{tabular}
\end{table}

\begin{table}[H]
	\centering
	\caption{Prognozowany ruch do Internetu z posiadanych przez firmę serwerów internetowych}
	%\label{my-label}
	\begin{tabular}{|l|c|c|c|}
		\hline
		\multicolumn{1}{|c|}{\textbf{Serwery internetowe}} & Do internetu & Z internetu & \begin{tabular}[c]{@{}c@{}}Liczba jednoczesnych\\ sesji\end{tabular} \\ \hline
		\textbf{Serwer WWW}                                & 140          & 35          & 48                                                                   \\ \hline
		\textbf{Serwer FTP}                                & 380          & 70          & 16                                                                   \\ \hline
	\end{tabular}
\end{table}

\begin{table}[H]
	\centering
	\caption{Wymagania dotyczące przepływów generowanych przez aplikacje użytkownika}
	%\label{my-label}
	\begin{tabular}{|l|c|c|c|c|c|c|c|c|}
		\hline
		\multicolumn{1}{|c|}{\textbf{\begin{tabular}[c]{@{}c@{}}Grupa robocza\\ lub aplikacja\end{tabular}}} & \multicolumn{2}{c|}{Przeglądarka} & \multicolumn{2}{c|}{Wideokonferencja} & \multicolumn{2}{c|}{VoIP} & \multicolumn{2}{c|}{Klient FTP} \\ \hline
		\textbf{Konstruktorzy}                                                                               & 0               & 0               & 40                & 40                & 0           & 0           & 0              & 0              \\ \hline
		\textbf{Architekci}                                                                                  & 0               & 0               & 0                 & 0                 & 20          & 20          & 57             & 12             \\ \hline
		\textbf{Projektanci}                                                                                 & 67              & 10              & 0                 & 0                 & 20          & 20          & 72             & 10             \\ \hline
		\textbf{Zarząd}                                                                                      & 66              & 10              & 40                & 40                & 20          & 20          & 63             & 20             \\ \hline
		\textbf{WiFi}                                                                                        & 18              & 10              & 0                 & 0                 & 20          & 20          & 57             & 18             \\ \hline
	\end{tabular}
\end{table}

%----------------------------------------------------------------------------------------
%	SECTION 4
%----------------------------------------------------------------------------------------
\newpage
\section{Założenia projektowe}

%----------------------------------------------------------------------------------------
%	SECTION 5
%----------------------------------------------------------------------------------------
\newpage
\section{Projekt sieci}

\subsection{Projekt logiczny sieci}

\subsection{Wybór urządzeń sieciowych}

\subsection{Projekt adresacji IP}

\subsection{Projekt konfiguracji urządzeń}

\subsection{Projekt podłączenia do Internetu}

\subsection{Analiza bezpieczeństwa i niezawodności sieci}

\subsection{Kosztorys}

%----------------------------------------------------------------------------------------
%	SECTION 6
%----------------------------------------------------------------------------------------
\newpage
\section{Karty katalogowe proponowanych urządzeń}

%----------------------------------------------------------------------------------------
%	BIBLIOGRAPHY
%----------------------------------------------------------------------------------------
%\newpage
%\bibliographystyle{apalike}
%\begin{thebibliography}{9}
%
%\bibitem{wikipediacluster} 
%Wikipedia: Computer cluster,
%\\\texttt{https://en.wikipedia.org/wiki/Computer\_cluster}
%	
%\bibitem{wikipediasieve} 
%Wikipedia: Sieve of Erastosthenes,
%\\\texttt{https://en.wikipedia.org/wiki/Sieve\_of\_Eratosthenes}
%
%\end{thebibliography}

%----------------------------------------------------------------------------------------
\end{document}